
\begin{abstract}
    Le nuove tecnologie di sequenziamento producono dati che hanno caratteristiche diverse da quelle che hanno dominato gli ultimi 10-15 anni. 
    In particolare ad oggi vengono prodotte read molto più lunghe (da 10 a 500 volte più lunghe) ma con un tasso di errore più elevato. Ciò richiede di rivedere gli approcci sviluppati in precedenza. 
    Questo stage affronta un importante problema in pangenomica computazionale: ottenere un grafo di variazioni a partire da un allineamento multiplo di long read. 
    Più precisamente, lo stage studierà il problema nel caso ristretto di grafo ottenuto a partire da un insieme di trascritti (grafo di splicing).
\end{abstract}

