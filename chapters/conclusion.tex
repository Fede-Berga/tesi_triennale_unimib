Lo Splicing Graph aciclico, descritto in questa relazione, è un caso ristretto del problema di costruzione di uno splicing graph, infatti si potrebbe pensare ad introdurre cicli nel grafo, in modo da evidenziare ancora meglio le omologie tra i trascritti allineati.

L'introduzione dei cicli introduce una serie di problematiche che devono essere sviscerate e analizzate, tra tutte:

\begin{itemize}
    \item \textsc{Convenienza nella formazione di cicli}: Nel momento in cui sono presenti sezioni duplicate nel grafo, si potrebbe pensare di inserire un ciclo, ma sezioni troppo corte produrrebbero un grafo con troppi cicli, che non evidenzierebbe al meglio le omologie tra le sequenze in input; si pensi ad esempio ad un grafo formato dai soli nodi corrispondenti ai simboli dell'alfabeto su cui sono costruite le sequenze, è chiaro che sarebbe poco utile e non congeniale agli scopi per cui è creato.
    D'altro canto, scegliere solo grossi frammenti per effettuare i cicli non evidenzierebbe correttamente le omologie presenti nei trascritti.
    \item \textsc{Criteri di ottimalità del grafo}: Quale criterio impostare per la "ricerca" di un buon grafo non è un problema semplice da risolvere, di certo non è possibile utilizzare i criteri di bontà descritti nei capitoli precedenti, in quanto, introducendo cicli, il numero di percorsi $first\_node \to last\_node$ sarebbero infiniti.
\end{itemize}

\newpage

\section{Sviluppi Futuri}

In conclusione credo di aver preso parte ad un'esperienza molto formativa che mi ha insegnato ad affrontare le difficoltà che potrebbero presentarsi in questo lavoro.

Durante lo stage ho imparato ad utilizzare un nuovo linguaggio, all'inizio astruso, ma in corso d'opera piacevole da utilizzare.
Ho imparato ad interfacciami con Git e Github per la manutenzione del progetto.

In generale ho affinato le conoscenze acquisite durante i precedenti tre anni di studio, ad esempio le conoscenze di analisi e progettazione del software, di algoritmi, è stato in generale uno stage formativo e consigliato.

Non sono certo di continuare il mio percorso nell'ambito della bioinformatica, certamente è stimolante, ma credo di conoscere ancora troppo poco di questo grande mondo.

Spero, alla fine della laure magistrale, di avere un'idea d'insieme più chiara di cosa è l'informatica e di cosa può offrire, in modo da scegliere correttamente cosa fare nella vita.