Questa tesi si prefigge l'obiettivo di illustrare ed argomentare l'esperienza di stage maturata durante il terzo anno di laurea triennale nella sua completezza. Verrà posta particolare enfasi sulle motivazioni che mi hanno portato a scegliere questo stage, le tecnologie utilizzate, i concetti teorici alla base, i problemi affrontati con le relative soluzioni e le nozioni da me apprese, oltre che un'introduzione ai concetti di biologia necessari al fine di comprendere al meglio questo elaborato.

\section{Stage in Bioinformatica}
Al terzo anno della laurea in Scienze Informatiche dell'università di Milano Bicocca, è chiesto agli studenti di affrontare uno stage su argomenti da loro scelti, al fine di approfondire le conoscenze in materia e acquisire competenze professionalizzanti.

Avendo frequentato l'insegnamento facoltativo di Bioinformatica durante il primo semestre del terzo anno, ho avuto la possibilità di appassionarmi alla materia. Nonostante all'inizio la trovassi astrusa, si è rivelata affascinante e mi ha fornito la conoscenza necessaria per portare a temine al meglio l'esperienza di tirocinio.

Ciò mi ha portato a scegliere uno stage difficoltoso, ma stimolante e molto formativo, che affronta un importante problema della pangenomica computazionale, cioè la costruzione, in linguaggio Rust, di uno splicing graph a partire da un allineamento multiplo di trascritti in input.

\newpage

\section{Struttura della Tesi}
Questa tesi è suddivisa in 5 capitoli.

\begin{itemize}
    \item \textsc{Introduzione : } il capitolo che spiega lo stage svolto e le motivazioni che mi hanno portato a sceglierlo.
    \item \textsc{Biologia Comutazionale : } il capitolo che introduce il lettore alla biologia computazionale, fornendone una descrizione storica, storica, sulle metodologie utilizzate e sugli obiettivi perseguiti.
    \item \textsc{Problema della Costruzione dello Splicing Graph : } il capitolo che introduce al problema e formalizza lo Splicing Graph e descrive ad alcuni concetti teorici preliminari, quali allineamento semplice e multiplo di sequenze.
    \item \textsc{Soluzione Proposta : } il capitolo che spiega i vari passaggi che mi hanno portato all'elaborazione della soluzione proposta, presenta i concetti teorici alla base e descrive l'architettura del software creato.
    \item \textsc{Conclusioni e sviluppi futuri : } il capitolo che introduce allo splicing graph ciclico e descrive gli sviluppi futuri.
\end{itemize}