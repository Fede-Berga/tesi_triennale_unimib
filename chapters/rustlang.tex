\section{Cos'è Rust?}
%mettere cappello introduttivo
%perchè rust
\label{section:rust_intro}

\begin{figure}[ht]
    \centering
    \includegraphics[scale= 0.5]{images/rustlang.jpg}
    \caption{Logo del Linguaggio Rust affiancato a quello di Mozilla}
    \label{fig:rustlang}
\end{figure}

Rust è un linguaggio di programmazione compilato, sviluppato da Mozilla Research in collaborazione con la comunità open source, perseguendo obiettivi di efficienza e sicurezza e idoneo a sviluppare software di sistema concorrente. \\
Rust è sintatticamente simile al C++, ma a differenza di quest'ultimo, utilizza un meccanismo chiamato borrow checker che permette di salvaguardare la memoria senza ricorrere a \textit{gabrbage collection} o \textit{reference counting}(Che rimane una feature opzionale del linguaggio)\cite{10.5555/3271463}.

\section{Storia del Linguaggio}
Rust nasce da un progetto personale di \textit{Graydon Hoare}, un dipendente di Mozilla. Successivamente il progetto viene ampliato fino al suo annuncio nel 2010.
Lo stesso anno iniziarono a scrivere il compilatore \textit{rustc} che venne compilato per la prima volta nel 2011.

La pre-alpha release del linguaggio fu rilasciata del 2012, mentre la prima versione stabile, denominata Rust 1.0, fu distribuita del 2015. Dopo la vesione 1.0, release stabili sono rilasciate ogni 6 settimane \cite{rusthistory}.

Nell'agosto del 2020, mozilla dovette licenziare 250 dei suoi 1000 dipendenti, a causa degli effetti perniciosi della pandemia di SARS-COV-2. Molti di questi erano parte del team che si occupava dello sviluppo e mantenimento di \textit{Rust}. Ciò gettò delle perplessità sul progetto.
Nelle settima seguenti, il team di sviluppo di \textit{Rust} ammise il severo impatto dovuto alla pandemia, inoltre annunciò che stava lavorando per la formazione della Rust Foundation\cite{rust_is_in_trouble}.

In data 8 febbraio 2021 fu annunciata la formazione della Rust Foundation che include cinque compagnie fondatrici (AWS, Huawei, Google, Microsoft e Mozilla).

In data 6 Aprile 2021, Google ha annunciato supporto software per Rust nell'ambito di Android Open Source Project come alternativa a C/C++.

\section{Caratteristiche del Linguaggio}
Come accennato nella sezione~\ref{section:rust_intro}, il linguaggio \textit{Rust} è stato sviluppato per la programmazione di sistemi concorrenti. Per fare in modo di facilitarne l'implementazione e il mantenimento sono stati aggiunti meccanismi per la gestione sicura della memoria.

\clearpage

\subsection{NULL pointers e Dangling References}

\textit{Rust} non permette l'utilizzo di puntatori a \textbf{NULL} e non permette, ad esempio ad una funzione, di restituire reference a variabili locali. Ogni reference restituito deve essere validato da un lifetime. I lifetime vengono controllati a tempo di compilazione attraverso il \textit{Borrow Checker} \cite{10.5555/3271463}.\\

\begin{lstlisting}[language=C++, caption={Esempio di dangling reference C++}, captionpos=b]
#include<cstring>

&std::string dangling_function() {
    std::string s = "Hello World";
    return s;
}
\end{lstlisting}

\subsection{Gestione della Memoria}
Rust non implementa meccanismi di \textit{Garbage Collection}. Piuttosto le risorse sono gestite tramite il pattern \textbf{RAII} (Resource Acquisition Is Initialization).

In \textit{Rust}, come in C++, esiste il concetto di \textit{Reference} (usando il simbolo \&), che a differenza di quest'ultimo vengono chiamati \textit{Borrow}. 

La sicurezza dei \textit{Borrow} è verificata a tempo di compilazione dal \textit{Borrow Checker} che reviene comportamenti indefiniti.

Esistono ue tipologie di borrow, immutabili (\&T) e mutabili (\&mut T). Possono esistere più borrow immutabili ad un certo valore in certo scope, ma solo un borrow mutabile \cite{10.5555/3271463}.

\subsection{Ownership System}
In \textit{Rust} le risorse hanno un singolo \textit{Owner} (propritario) e sono rilasciate nel momento in cui il questul'ultimo esce di scope. Quindi ogni risorsa ha lo stesso scope del proprio owner.

In \textit{Rust} i valori possono essere passati per reference immutabile o mutabile, ricordando che possono esistere o più reference immutabili o un solo reference mutabile, ma anche per valore, in questo caso può avvenire una \textit{Move}. Per evitare \textit{Dangling References} l'owner originale non è più utilizzabile \cite{10.5555/3271463}.\\

\begin{lstlisting}[caption={Esempio di Move. s1 non può più essere utilizzato, il nuovo Owner della stringa è s2.}, captionpos=b]

fn main() {
    let s1 = String::from("hello");
    let s2 = s1;

    println!("{}, world!", s1); //Compile time Error
}
\end{lstlisting}

\section{Usi del linguaggio Rust}
Date le sue caratteristiche di sicurezza e affidabilità, il linguaggio \textit{Rust} trova impiego in molti ambiti:

\begin{itemize}
    \item Alcuni conponenti di Mozilla Firefox sono scritti in Rust, tra cui \textit{Servo}, un motore di ricerca parallelo.
    \item I componenti di alcuni sistemi operativi sono scritti in Rust, tra cui:
    \begin{itemize}
        \item Google Fuchsia: un sistema operativo
        \item Stratis : un file manager
        \item Redox : un sistema operativo Unix-like
    \end{itemize}
    \item Alcuni componenti di Microsoft Azure sono scritti in Rust
\end{itemize}

Dal 2016 al 2021 Rust è stato il lingaggio di programmazione più amato nella classifica stilata da Stack Overflow.









